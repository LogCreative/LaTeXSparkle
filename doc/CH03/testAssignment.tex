\documentclass[a4paper,12pt]{article}
\usepackage{CJKutf8}
\usepackage{amsthm}
\usepackage{amsmath}
\usepackage{amssymb}
\usepackage{geometry}

% 边距
\geometry{left=3.0cm,right=2.0cm,top=3.0cm,bottom=3.0cm}

% 大题
\newenvironment{firstlayer}{\begin{list}{}{\renewcommand{\makelabel}[1]{\textbf{##1}.\hfil}}}{\end{list}}

% 小题
\newenvironment{secondlayer}{\begin{list}{}{\renewcommand{\makelabel}[1]{(##1)\hfil}}}{\end{list}}

% 证明
\renewcommand{\proofname}{\textbf{证明}}

% 解
\providecommand{\sol}{\textbf{解}.~}

% 标题
\title{示例作业}
\author{LogCreative}
\date{2020 年 8 月}

\begin{document}
\begin{CJK}{UTF8}{song}
\maketitle

\begin{firstlayer}

\item[2] 简单图 $G$ 中,如果 $m>\frac{1}{2}(n-1)(n-2)$,证明 $G$ 不存在孤立节点。

\begin{proof}
    % 证明环境
    若不然,则有一孤立点 $v$,子图 $G'=G-v$ 的边数

    \begin{equation}\label{ls}
        \left\lvert E(G')\right\rvert = \left\lvert E(G)\right\rvert = m>\frac{1}{2}(n-1)(n-2) 
    \end{equation}

    然而,$G'$ 边数最多的情况是完全图 $K_{n-1}$,也就是
    \begin{equation}\label{lss}
        \left\lvert E(G')\right\rvert \leq \frac{1}{2}(n-1)(n-2)
    \end{equation}

    这与式子 \eqref{ls} 矛盾。

    由于当孤立节点数不止一个时,式子 \eqref{lss} 依然成立。所以不存在孤立节点。

\end{proof}

\begin{secondlayer}

    \item[2] \sol 这是一个解。

\end{secondlayer}

\end{firstlayer}

\end{CJK}
\end{document}