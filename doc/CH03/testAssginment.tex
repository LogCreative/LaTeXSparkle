\documentclass[a4paper,12pt]{article}
\usepackage{CJKutf8}
\usepackage{amsmath}
\usepackage{geometry}
\usepackage{enumerate}
\usepackage{enumitem}
\usepackage{amsthm}
\usepackage{amssymb}
\setlist[enumerate,1]{font=\bfseries}
\geometry{left=3.0cm,right=2.0cm,top=3.0cm,bottom=3.0cm}

\newenvironment{firstlayer}%
{\begin{list}{}{\renewcommand{\makelabel}[1]{\textbf{##1}.\hfil}
}}
{\end{list}}
\newenvironment{secondlayer}%
{\begin{list}{}{\renewcommand{\makelabel}[1]{(##1)\hfil}
}}
{\end{list}}

\renewcommand{\proofname}{\textbf{证明}}

\providecommand{\sol}{\textbf{解}.~}

\title{示例作业}
\author{LogCreative}
\begin{document}

\begin{CJK}{UTF8}{song}

\maketitle

\begin{firstlayer}
  \item[2]简单图 $G$ 中,如果 $m>\frac{1}{2}(n-1)(n-2)$,证明 $G$ 不存在孤立节点。
  \begin{proof}
    若不然,则有一孤立点 $v$,子图 $G'=G-v$ 的边数
    \begin{equation}\label{ls}
      \left\lvert E(G') \right\rvert =\left\lvert E(G)\right\rvert =m>\frac{1}{2}(n-1)(n-2)
    \end{equation}
      
    然而,$G'$ 边数最多的情况是完全图 $K_{n-1}$,也就是
    \begin{equation}\label{lss}
      \left\lvert E(G') \right\rvert  \leq \frac{1}{2}(n-1)(n-2)
    \end{equation}

    这与式子 (\ref{ls}) 矛盾。

    由于当孤立节点数不止一个时,式子 (\ref{lss}) 依然成立。所以不存在孤立节点。

  \end{proof}
\end{firstlayer}

\end{CJK}

\end{document}