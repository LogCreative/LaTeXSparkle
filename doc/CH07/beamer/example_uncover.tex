\documentclass{beamer}
\mode<presentation>
\usetheme{CambridgeUS}
\usecolortheme{crane}
\usefonttheme{serif}
\usepackage{CJKutf8}
\hypersetup{unicode}
\usepackage{bookmark}

% To ignore the warning of "pdfauthor"
\makeatletter
\def\Hy@WarnOptionDisabled#1{
    \def\next{#1}%
    \def\ignore{pdfauthor}%
    \ifx\next\ignore%
    \else\Hy@Warning{%
        Option `#1' has already been used,\MessageBreak setting the option has no effect%
    }\fi
}
\makeatother

\begin{document}
\begin{CJK}{UTF8}{hei}
    \begin{frame}
        \frametitle{列表环境}
        \begin{itemize}[<+->]
            \item 第 1 点
            \item 第 2 点
            \item 第 3 点
        \end{itemize}
    \end{frame}

    \begin{frame}
        \frametitle{悬念式展开}
        第一行推导

        \begin{actionenv}<2->
            第二行推导
        \end{actionenv}

        \begin{block}<3->{结论}
            我的结论
        \end{block}
    \end{frame}

    \begin{frame}<presentation>
        \frametitle{幻灯片与讲义分离}
        只出现在幻灯片中。
    \end{frame}

    \begin{frame}<all>
        \frametitle{幻灯片与讲义分离}
        幻灯片和文稿都会出现。

        \only<article>{只会出现在文稿中。}
    \end{frame}

    \begin{frame}
        \frametitle{突出条目}
        \begin{itemize}[<+-| alert@+>]
            \item 第 1 点
            \item 第 2 点
            \item 第 3 点
        \end{itemize}
    \end{frame}

\end{CJK}
\end{document}
